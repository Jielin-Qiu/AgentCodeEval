\documentclass{article}
\usepackage[utf8]{inputenc}
\usepackage{amsmath}
\usepackage{amsfonts}
\usepackage{amssymb}
\usepackage{graphicx}
\usepackage{hyperref}
\usepackage{enumitem}
\usepackage{booktabs}
\usepackage{geometry}
\geometry{margin=1in}

\title{AgentCodeEval: A Novel Benchmark for Evaluating Long-Context Language Models in Software Development Agent Tasks}
\author{Research Proposal}
\date{\today}

\begin{document}

\maketitle

\begin{abstract}
We propose AgentCodeEval, a comprehensive benchmark for evaluating long-context language models (LLMs) in software development agent scenarios. Unlike existing code evaluation benchmarks that focus on single-function completion or isolated code understanding, AgentCodeEval addresses the critical gap in evaluating LLMs' ability to perform complex, multi-file software development tasks that require understanding large codebases, managing dependencies, and implementing features across multiple sessions. This benchmark targets real-world software development workflows where agents must navigate complex file structures, understand inter-module dependencies, and maintain architectural consistency across large-scale codebases.
\end{abstract}

\section{Introduction}

\subsection{Motivation}
Current long-context LLM evaluation benchmarks primarily focus on:
\begin{itemize}
    \item Single-function code completion (HumanEval, MBPP)
    \item Document retrieval and question-answering
    \item Synthetic needle-in-haystack tasks
    \item Mathematical reasoning over long sequences
\end{itemize}

However, these benchmarks fail to capture the complexity of real-world software development agent tasks, which require:
\begin{itemize}
    \item Multi-file codebase understanding
    \item Cross-module dependency management
    \item Architectural pattern recognition and maintenance
    \item Multi-session development workflows
    \item Integration of new features into existing systems
\end{itemize}

\subsection{Research Gap}
No existing benchmark comprehensively evaluates long-context LLMs in the software development agent domain, despite the growing importance of AI-powered development tools.

\section{AgentCodeEval: Benchmark Design}

\subsection{Core Concept}
AgentCodeEval evaluates LLMs on realistic software development scenarios that span multiple files, require understanding of complex dependencies, and simulate multi-session development workflows.

\subsection{Key Design Principles}
\begin{enumerate}
    \item \textbf{Real-world Relevance}: Tasks mirror actual software development workflows
    \item \textbf{Multi-file Complexity}: Each task involves 5-50+ files with interdependencies
    \item \textbf{Long-context Requirement}: Tasks require understanding of 10K-100K+ tokens
    \item \textbf{Agent-specific Evaluation}: Focus on development agent capabilities rather than code completion
    \item \textbf{Scalable Generation}: Automated pipeline for creating new evaluation instances
\end{enumerate}

\subsection{Task Categories}

\subsubsection{Cross-file Refactoring}
\begin{itemize}
    \item Extract functionality into new modules while maintaining dependencies
    \item Rename classes/functions across multiple files
    \item Restructure package hierarchies
    \item Apply design patterns to existing code
\end{itemize}

\subsubsection{Feature Implementation}
\begin{itemize}
    \item Add authentication systems across web applications
    \item Implement new API endpoints with proper routing and validation
    \item Integrate new libraries while maintaining compatibility
    \item Add logging and monitoring to existing systems
\end{itemize}

\subsubsection{Bug Investigation and Fixing}
\begin{itemize}
    \item Trace errors through complex call stacks
    \item Identify and fix race conditions in concurrent code
    \item Debug performance issues across multiple modules
    \item Fix integration problems between components
\end{itemize}

\subsubsection{Architecture Understanding and Documentation}
\begin{itemize}
    \item Generate comprehensive documentation for complex modules
    \item Create architectural diagrams from code analysis
    \item Identify code smells and suggest improvements
    \item Explain design decisions and trade-offs
\end{itemize}

\subsection{Difficulty Levels}
\begin{enumerate}
    \item \textbf{Easy (2-5 files)}: Simple cross-file operations, basic dependency understanding
    \item \textbf{Medium (5-15 files)}: Moderate complexity with multiple dependencies
    \item \textbf{Hard (15-50+ files)}: Complex architectural understanding, deep dependency analysis
\end{enumerate}

\section{Data Sources and Resources}

\subsection{Foundation Datasets}
\begin{table}[h]
\centering
\begin{tabular}{@{}lll@{}}
\toprule
Dataset & Size & Description \\
\midrule
The Stack (BigCode) & 6TB & Permissively-licensed code in 358+ languages \\
Project CodeNet (IBM) & 14M samples & Problem-solution pairs across 4K problems \\
CodeSearchNet & 2M pairs & Comment-code pairs in 6 languages \\
\bottomrule
\end{tabular}
\caption{Primary data sources for AgentCodeEval}
\end{table}

\subsection{Analysis Tools}
\begin{itemize}
    \item \textbf{Tree-sitter}: Multi-language incremental parser for AST analysis
    \item \textbf{dep-tree}: Codebase complexity and dependency visualization
    \item \textbf{Sokrates}: Polyglot source code examination framework
\end{itemize}

\subsection{Seed Repositories}
High-quality, well-documented projects for task generation:
\begin{itemize}
    \item TensorFlow (191k stars): Complex ML framework
    \item PyTorch (92k stars): Well-structured deep learning library
    \item scikit-learn: Exemplary Python project structure
    \item Keras: Clean, well-documented codebase
    \item React.js: Modern frontend framework
    \item Node.js/Express: Backend development patterns
\end{itemize}

\section{Methodology}

\subsection{Task Generation Pipeline}

\subsubsection{Phase 1: Codebase Analysis}
\begin{enumerate}
    \item Parse codebases using Tree-sitter for AST generation
    \item Analyze dependencies using dep-tree and custom tooling
    \item Identify architectural patterns and design principles
    \item Extract meaningful code segments and their relationships
\end{enumerate}

\subsubsection{Phase 2: Scenario Creation}
\begin{enumerate}
    \item Generate realistic development scenarios based on codebase analysis
    \item Create multi-file tasks with clear objectives and constraints
    \item Ensure tasks require long-context understanding
    \item Validate task complexity and solvability
\end{enumerate}

\subsubsection{Phase 3: Reference Solution Generation}
\begin{enumerate}
    \item Use state-of-the-art LLMs (Claude, GPT-4) to create reference implementations
    \item Manual validation and refinement of solutions
    \item Multiple solution variants for comprehensive evaluation
    \item Documentation of expected approaches and trade-offs
\end{enumerate}

\subsection{Evaluation Metrics}

\subsubsection{Functional Correctness}
\begin{itemize}
    \item Code compiles and runs without errors
    \item Unit tests pass (where applicable)
    \item Integration tests validate cross-module functionality
    \item Performance benchmarks (for optimization tasks)
\end{itemize}

\subsubsection{Architectural Integrity}
\begin{itemize}
    \item Maintains existing design patterns
    \item Preserves separation of concerns
    \item Follows established coding conventions
    \item Respects dependency injection patterns
\end{itemize}

\subsubsection{Dependency Management}
\begin{itemize}
    \item All imports/references are correct and minimal
    \item No circular dependencies introduced
    \item Proper abstraction levels maintained
    \item Interface contracts preserved
\end{itemize}

\subsubsection{Code Quality}
\begin{itemize}
    \item Readability and maintainability scores
    \item Documentation completeness and quality
    \item Error handling and edge case coverage
    \item Security best practices adherence
\end{itemize}

\section{Development Plan}

\subsection{Phase 1: Foundation Setup (Weeks 1-2)}
\begin{itemize}
    \item Download and preprocess The Stack dataset (Python, JavaScript, TypeScript, Java)
    \item Set up analysis tools (Tree-sitter, dep-tree, Sokrates)
    \item Select and analyze 10-15 seed repositories
    \item Develop initial codebase parsing and analysis pipeline
\end{itemize}

\subsection{Phase 2: Task Generation Pipeline (Weeks 3-4)}
\begin{itemize}
    \item Implement automated dependency analysis
    \item Create task template system for different categories
    \item Develop scenario generation algorithms
    \item Build validation framework for task quality
\end{itemize}

\subsection{Phase 3: Reference Solution Creation (Weeks 5-6)}
\begin{itemize}
    \item Integrate Claude/GPT-4 APIs for solution generation
    \item Implement solution validation and testing framework
    \item Create multiple solution variants for each task
    \item Develop automated quality assessment tools
\end{itemize}

\subsection{Phase 4: Benchmark Assembly (Weeks 7-8)}
\begin{itemize}
    \item Generate 100-500 evaluation instances across difficulty levels
    \item Implement comprehensive evaluation metrics
    \item Create baseline evaluations with existing LLMs
    \item Validate benchmark difficulty and discrimination
\end{itemize}

\subsection{Phase 5: Validation and Publication (Weeks 9-10)}
\begin{itemize}
    \item Conduct thorough benchmark validation
    \item Evaluate multiple state-of-the-art LLMs
    \item Prepare research paper and documentation
    \item Release benchmark and evaluation framework
\end{itemize}

\section{Expected Contributions}

\subsection{Research Contributions}
\begin{enumerate}
    \item First comprehensive benchmark for long-context LLMs in software development agent tasks
    \item Novel evaluation framework for multi-file code understanding and generation
    \item Automated pipeline for generating realistic software development scenarios
    \item Comprehensive analysis of current LLM capabilities in agent-based development workflows
\end{enumerate}

\subsection{Practical Impact}
\begin{enumerate}
    \item Enable rigorous evaluation of software development AI agents
    \item Provide clear metrics for progress in long-context code understanding
    \item Guide development of more effective coding assistants
    \item Establish standards for evaluating AI-powered development tools
\end{enumerate}

\section{Resource Requirements}

\subsection{Computational Resources}
\begin{itemize}
    \item Standard server with 32GB+ RAM (no GPU required for analysis)
    \item 100GB+ storage for datasets and generated tasks
    \item API access to Claude/GPT-4/Gemini for reference generation
\end{itemize}

\subsection{No Requirements}
\begin{itemize}
    \item Human annotation or labeling
    \item GPU resources for training
    \item Complex distributed infrastructure
\end{itemize}

\section{Timeline and Deliverables}

\subsection{10-Week Timeline}
\begin{itemize}
    \item Weeks 1-2: Foundation setup and tool configuration
    \item Weeks 3-4: Task generation pipeline development
    \item Weeks 5-6: Reference solution creation and validation
    \item Weeks 7-8: Benchmark assembly and initial evaluation
    \item Weeks 9-10: Validation, documentation, and publication
\end{itemize}

\subsection{Key Deliverables}
\begin{enumerate}
    \item AgentCodeEval benchmark dataset (100-500 instances)
    \item Automated evaluation framework and metrics
    \item Baseline performance analysis of major LLMs
    \item Research paper documenting methodology and results
    \item Open-source release of tools and benchmark
\end{enumerate}

\section{Conclusion}

AgentCodeEval addresses a critical gap in long-context LLM evaluation by focusing on realistic software development agent tasks. By leveraging existing high-quality code datasets and automated analysis tools, we can create a comprehensive benchmark without requiring extensive human annotation or GPU resources. This benchmark will provide valuable insights into current LLM capabilities and guide future development of AI-powered software development tools.

The proposed approach is both novel and practical, offering a scalable methodology for generating realistic evaluation scenarios while maintaining rigorous evaluation standards. AgentCodeEval has the potential to become a standard benchmark in the field, similar to how HumanEval established standards for code completion evaluation.

\end{document}
